\documentclass{article}

\usepackage{amssymb}
\usepackage{algorithm}
\usepackage[noend]{algpseudocode}

\usepackage{biblatex}
\addbibresource{bibliography.bib}

\begin{document}

\section{Background}

Given data $(y_i, x_i, z_i), i = 1, \dots, n$ with outcome $y_i \in \mathbb{R}$, covariates $x_i \in \mathbb{R}^p$ and effect modifiers $z_i \in \mathbb{R}^q$ one assumes the varying coefficient model
%
\begin{equation}
y_i = \sum_{j=1}^p x_{ij} \beta_j(z_i) + \varepsilon_i = x_i^T \beta(z_i) + \varepsilon_i,
\end{equation}
%
\cite{hastie1993varying}. The coefficients $\beta(\cdot) = (\beta_1(\cdot), \dots, \beta_p(\cdot))^T$ determine a (rather simple) functional, relationship between outcome $y$ and (likely low dimensional) covariate $x$. The coefficients themselves are considered to be (potentially complex) functions of the (potentially high dimensional) effect modifier $z$.\\
Each varying coefficient mapping $\beta_j(\cdot)$ is estimated using an ensemble of gradient
boosted decision trees. This is done by iteratively minimizing a loss function
%
\begin{equation}
    L(y, \beta) = \sum_{i=1}^n l(y_i, x_i^T\beta(z_i)).
\end{equation}
%
The generic gradient boosting algorithm described in \cite{friedman2001greedy}. Is adapted for the varying coefficient scenario. The resulting procedure is sketched in Algorithm~\ref{alg:vcboost} below. A similar algorithm has already been described by \cite{zhou2019tree}.
%
\begin{algorithm}[h]
\caption{VCBoost}
\label{alg:vcboost}
	\begin{algorithmic}[1]
		\State $\beta_j^{(0)}(z) = 0$, $j = 1, \dots, p$
		\For{$m = 1, \dots$, maxiter}	
			\For{$j = 1, \dots, p$}
				\State $\tilde{y_i} = - \frac{\partial l(y_i, x_i^T\beta^{(m-1)}(z_i))}{\partial \beta_j^{(m-1)}(z_i)}$ \Comment{negative gradient wrt. $\beta_j$}
				\State $h = $ regression tree fit on $\{\tilde{y_i}, z_i\}_{i=1, \dots, n}$
				\State $\rho = $ argmin$_{\rho} \left[ \sum_{i=1}^n l (y, x_i^T\beta^{(m-1)}(z_i) + \rho h(z_i) x_{ij}) \right]$ \Comment{line search}
				\State $\beta_j^{(m)}(\cdot) = \beta_j^{(m-1)}(\cdot) + \rho h(\cdot)$
			\EndFor
		\EndFor
	\end{algorithmic}
\end{algorithm}\\
%
The pseudoresponses $\tilde{y_i}$ in Line 4 are obtained by noting that
%
\begin{equation}
\tilde{y_i} = - \frac{\partial l(y_i, x_i^T\beta^{(m-1)}(z_i))}{\partial \beta_j^{(m-1)}(z_i)} = - \frac{\partial l(y_i, x_i^T\beta^{(m-1)}(z_i))}{\partial x_i^T\beta^{(m-1)}(z_i)}x_{ij}
\end{equation}
%
and the derivative remaining on the right hand side is the usual gradient based on the previous fitted responses $x_i^T\beta^{(m-1)}(z_i)$. Hence, this is easily determined for common loss functions. Also this part of the pseudo response is the same, regardless of the coordinate ($j = 1, \dots, p$) currently handled in the inner loop. A variation of the algorithm only computes $\frac{\partial l(y_i, x_i^T\beta^{(m-1)}(z_i))}{\partial x_i^T\beta^{(m-1)}(z_i)}$ once during the outer loop and then updates all coefficient estimates before recomputing.\\
Also the line search determining the step size $\rho$ can be done globally, e.g. determining one $\rho$ based on all observations $i = 1, \dots n$ as shown above. Alternatively, the line search can be done separately for each leaf in the regression tree $h$ (see Eq. (18) in \cite{friedman2001greedy}).\\
As usual one can also introduce a learning rate $ < 1$ to shrink the step size for the update in row 7 of the algorithm.\\
Finally there would also be many hyper-parameters controlling the tree building process in line 5.

\subsection{Details for specific loss functions}
...

\printbibliography

\end{document}